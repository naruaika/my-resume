%%%%%%%%%%%%%%%%%%%%%%%%%%%%%%%%%%%%%%%%%
% Awesome Cover Letter
% XeLaTeX Template
% Version 1.3 (30/3/2020)
%
% This template has been downloaded from:
% http://www.LaTeXTemplates.com
%
% Original authors:
% Claud D. Park (posquit0.bj@gmail.com)
% Lars Richter (mail@ayeks.de)
% With modifications by:
% Vel (vel@latextemplates.com)
%
% License:
% CC BY-NC-SA 3.0 (http://creativecommons.org/licenses/by-nc-sa/3.0/)
%
% Important note:
% This template must be compiled with XeLaTeX, the below lines will ensure this
%!TEX TS-program = xelatex
%!TEX encoding = UTF-8 Unicode
%
%%%%%%%%%%%%%%%%%%%%%%%%%%%%%%%%%%%%%%%%%

%----------------------------------------------------------------------------------------
%	PACKAGES AND OTHER DOCUMENT CONFIGURATIONS
%----------------------------------------------------------------------------------------

\documentclass[11pt, a4paper]{awesome-cv} % A4 paper size by default, use 'letterpaper' for US letter

\geometry{left=2cm, top=1.5cm, right=2cm, bottom=2cm, footskip=.5cm} % Configure page margins with geometry

\fontdir[fonts/] % Specify the location of the included fonts

% Color for highlights
\colorlet{awesome}{awesome-red} % Default colors include: awesome-emerald, awesome-skyblue, awesome-red, awesome-pink, awesome-orange, awesome-nephritis, awesome-concrete, awesome-darknight
%\definecolor{awesome}{HTML}{CA63A8} % Uncomment if you would like to specify your own color

% Colors for text - uncomment and modify
%\definecolor{darktext}{HTML}{414141}
%\definecolor{text}{HTML}{414141}
%\definecolor{graytext}{HTML}{414141}
%\definecolor{lighttext}{HTML}{414141}

\renewcommand{\acvHeaderSocialSep}{\quad\textbar\quad} % If you would like to change the social information separator from a pipe (|) to something else

%----------------------------------------------------------------------------------------
%	PERSONAL INFORMATION
%	Comment any of the lines below if they are not required
%----------------------------------------------------------------------------------------

\name{Naufan Rusyda}{Faikar}
\address{Jalan Inten Simbarjo IV, Blok H4-3, Kompleks Cipageran Asri, Kelurahan Cipageran, Kecamatan Cimahi Utara, Kota Cimahi, Provinsi Jawa Barat, Indonesia}
\mobile{(+62) 813-1322-2479}

\email{idmuslimdev@gmail.com}
\github{naruaika}
\linkedin{naufan-rusyda-faikar}
\stackoverflow{8791891}{naufan-rusyda-faikar}
\homepage{dev.to/naruaika}

\position{Software Engineer} % Your expertise/fields
\quote{``Work done from the heart will bring out satisfaction in people’s hearts.''} % A quote or statement

%----------------------------------------------------------------------------------------
%	RECIPIENT/POSITION/LETTER INFORMATION
%	All of the below lines must be filled out
%----------------------------------------------------------------------------------------

\recipient{Tim Rekrutmen Perusahaan}{PT Rekadia Solusi Teknologi\\Jakarta Selatan} % The company being applied to

\letterdate{\today} % The date on the letter, default is the date of compilation

\lettertitle{Lamaran Pekerjaan untuk \textit{Full Stack Programmer}} % The title of the letter

\letteropening{Yth. Ibu/Bapak,} % How the letter is opened

\letterclosing{Hormat Saya,} % How the letter is closed

\letterenclosure[Terlampir]{Curriculum Vitae} % Any enclosures with the letter

\makecvfooter{\today}{Naufan Rusyda Faikar~~~·~~~Surat Pengantar}{} % Specify the letter footer with 3 arguments: (<left>, <center>, <right>), leave any of these blank if they are not needed

%----------------------------------------------------------------------------------------

\begin{document}

\makecvheader % Print the header

\makelettertitle % Print the title

%----------------------------------------------------------------------------------------
%	LETTER CONTENT
%----------------------------------------------------------------------------------------

\begin{cvletter}

%------------------------------------------------

\lettersection{Tentang Saya}

\hyphenation{per-kem-ba-ngan}

Saya seorang insinyur perangkat lunak profesional dengan satu tahun lebih tiga bulan pengalaman di industri dan empat tahun pengalaman di komunitas. Spesialisasi saya adalah teknologi web, tetapi saya menikmati pekerjaan yang berhubungan dengan matematika. Di waktu luang saya, saya suka mengeksplorasi hal-hal terkait perkembangan teknologi terbaru terutama di bidang kecerdasan buatan; menulis blog tentang hal-hal tersebut; mendengarkan siniar mengenai pendidikan, perbaikan diri, dan bisnis; serta terlibat cukup aktif di dalam komunitas maya menemani orang lain menuju solusi mereka sendiri. Motivasi terbesar saya adalah bahwa pekerjaan yang dilakukan dari hati akan memunculkan kepuasan di hati orang-orang.

%------------------------------------------------

\lettersection{Mengapa PT Rekadia Solusi Teknologi?}

Berangkat dari pengalaman profesional saya di industri teknologi alih daya selama setahun lebih tiga bulan, saya memahami betul misi PT Rekadia Solusi Teknologi dalam memastikan keberhasilan transformasi digital bisnis-bisnis pelanggan melalui teknologi komputerisasi.

Solusi digital akan selalu dibutuhkan untuk membantu strategi bisnis mencapai tujuan-tujuannya. Terutama pada era di mana kolaborasi antar teknologi memungkinkan para pelaku bisnis untuk mendapatkan berbagai kemudahan dalam berinovasi. Solusi digital adalah sebuah keniscayaan, namun solusi digital yang efektif memerlukan kemampuan dan pengalaman sebanyak yang dimiliki oleh PT Rekadia Solusi Teknologi.

%------------------------------------------------

\lettersection{Mengapa Saya?}

Saya bekerja sebagai back-end developer selama kurang lebih setahun, namun kemudian dalam tiga bulan terakhir ini saya bekerja sebagai \textit{full-stack developer}. Selama kurun waktu itu pula, saya beberapa kali bekerja sebagai front-end developer dalam proyek pengembangan halaman gim dan web interaktif.

Selama bekerja di perusahaan saat ini, saya hanya memiliki dua klien. Namun, mereka adalah klien yang sangat setia. Ketika surat ini dibuat, klien pertama saya telah meminta saya secara khusus kepada atasan saya agar saya yang ditugaskan untuk mengerjakan proyek mereka yang selanjutnya. Sehingga jika ditindaklanjuti, proyek tersebut akan menjadi proyek saya yang keempat bersama mereka. Sementara itu, setelah proyek berjalan selama tiga bulan pertama, klien kedua saya menyatakan kepuasannya terhadap performa tim kami dalam testimonial yang dikirimkan melalui surel. Bekerja sama dengan tim kami telah mengobati keraguan pejabat eksekutif tertinggi di perusahaan klien terhadap pengalaman buruk mereka beberapa kali bekerja sama dengan perusahaan sejenis sebelumnya. Sebagai catatan, semua proyek saya tersebut dikerjakan dengan metode kerja jarak jauh sejak awal, sehingga diperlukan kemampuan komunikasi yang andal.

Pengalaman saya empat kali sebagai asisten dosen semasa kuliah dulu juga menyumbangkan banyak pengetahuan dan wawasan tentang bagaimana merumuskan pendekatan solusi atas berbagai persoalan komputasi dan bagaimana cara terbaik untuk menjelaskannya kepada orang lain.

Dengan demikian, saya memohon agar kecocokan saya dapat dipertimbangkan atas lowongan pekerjaan sebagai \textit{full-stack developer} di PT Rekadia Solusi Teknologi.

%------------------------------------------------

\end{cvletter}

%----------------------------------------------------------------------------------------

\makeletterclosing % Print the signature and enclosures

\end{document}
